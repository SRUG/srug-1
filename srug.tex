%%
%% presentation.tex
%%
%% Made by Jakub Kuźma
%% Login   <kuba@jah.pl>
%%
%% Started on  Fri Oct 24 19:44:04 2008 Jakub Kuźma
%% Last update Fri Oct 24 19:44:04 2008 Jakub Kuźma
%%

\documentclass[12t]{beamer}
\usepackage[utf8]{inputenc}
\usepackage{polski}
\usepackage[polish]{babel}
\usepackage{graphicx}
\usepackage{color}
%%
%% pygments.tex
%%
%% Made by Jakub Kuźma
%% Login   <kuba@ibm>
%%
%% Started on  Sat Oct 25 02:13:17 2008 Jakub Kuźma
%% Last update Sat Oct 25 02:13:17 2008 Jakub Kuźma
%%

\usepackage{fancyvrb}
\usepackage{color}

\newcommand\at{@}
\newcommand\lb{[}
\newcommand\rb{]}
\newcommand\PYbg[1]{\textcolor[rgb]{0.00,0.50,0.00}{\textbf{#1}}}
\newcommand\PYbf[1]{\textcolor[rgb]{0.73,0.40,0.53}{\textbf{#1}}}
\newcommand\PYbe[1]{\textcolor[rgb]{0.40,0.40,0.40}{#1}}
\newcommand\PYbd[1]{\textcolor[rgb]{0.73,0.13,0.13}{#1}}
\newcommand\PYbc[1]{\textcolor[rgb]{0.00,0.50,0.00}{\textbf{#1}}}
\newcommand\PYbb[1]{\textcolor[rgb]{0.40,0.40,0.40}{#1}}
\newcommand\PYba[1]{\textcolor[rgb]{0.00,0.00,0.50}{\textbf{#1}}}
\newcommand\PYaJ[1]{\textcolor[rgb]{0.73,0.13,0.13}{#1}}
\newcommand\PYaK[1]{\textcolor[rgb]{0.00,0.00,1.00}{#1}}
\newcommand\PYaH[1]{\fcolorbox[rgb]{1.00,0.00,0.00}{1,1,1}{#1}}
\newcommand\PYaI[1]{\textcolor[rgb]{0.69,0.00,0.25}{#1}}
\newcommand\PYaN[1]{\textcolor[rgb]{0.00,0.00,1.00}{\textbf{#1}}}
\newcommand\PYaO[1]{\textcolor[rgb]{0.00,0.00,0.50}{\textbf{#1}}}
\newcommand\PYaL[1]{\textcolor[rgb]{0.73,0.73,0.73}{#1}}
\newcommand\PYaM[1]{\textcolor[rgb]{0.74,0.48,0.00}{#1}}
\newcommand\PYaB[1]{\textcolor[rgb]{0.00,0.25,0.82}{#1}}
\newcommand\PYaC[1]{\textcolor[rgb]{0.67,0.13,1.00}{#1}}
\newcommand\PYaA[1]{\textcolor[rgb]{0.00,0.50,0.00}{#1}}
\newcommand\PYaF[1]{\textcolor[rgb]{1.00,0.00,0.00}{#1}}
\newcommand\PYaG[1]{\textcolor[rgb]{0.10,0.09,0.49}{#1}}
\newcommand\PYaD[1]{\textcolor[rgb]{0.25,0.50,0.50}{\textit{#1}}}
\newcommand\PYaE[1]{\textcolor[rgb]{0.63,0.00,0.00}{#1}}
\newcommand\PYaZ[1]{\textcolor[rgb]{0.00,0.50,0.00}{\textbf{#1}}}
\newcommand\PYaX[1]{\textcolor[rgb]{0.00,0.50,0.00}{#1}}
\newcommand\PYaY[1]{\textcolor[rgb]{0.73,0.13,0.13}{#1}}
\newcommand\PYaR[1]{\textcolor[rgb]{0.10,0.09,0.49}{#1}}
\newcommand\PYaS[1]{\textcolor[rgb]{0.25,0.50,0.50}{\textit{#1}}}
\newcommand\PYaP[1]{\textcolor[rgb]{0.49,0.56,0.16}{#1}}
\newcommand\PYaQ[1]{\textcolor[rgb]{0.40,0.40,0.40}{#1}}
\newcommand\PYaV[1]{\textcolor[rgb]{0.00,0.00,1.00}{\textbf{#1}}}
\newcommand\PYaW[1]{\textcolor[rgb]{0.73,0.13,0.13}{#1}}
\newcommand\PYaT[1]{\textcolor[rgb]{0.50,0.00,0.50}{\textbf{#1}}}
\newcommand\PYaU[1]{\textcolor[rgb]{0.82,0.25,0.23}{\textbf{#1}}}
\newcommand\PYaj[1]{\textcolor[rgb]{0.00,0.50,0.00}{#1}}
\newcommand\PYak[1]{\textcolor[rgb]{0.73,0.40,0.53}{#1}}
\newcommand\PYah[1]{\textcolor[rgb]{0.63,0.63,0.00}{#1}}
\newcommand\PYai[1]{\textcolor[rgb]{0.10,0.09,0.49}{#1}}
\newcommand\PYan[1]{\textcolor[rgb]{0.67,0.13,1.00}{\textbf{#1}}}
\newcommand\PYao[1]{\textcolor[rgb]{0.73,0.40,0.13}{\textbf{#1}}}
\newcommand\PYal[1]{\textcolor[rgb]{0.25,0.50,0.50}{\textit{#1}}}
\newcommand\PYam[1]{\textbf{#1}}
\newcommand\PYab[1]{\textit{#1}}
\newcommand\PYac[1]{\textcolor[rgb]{0.73,0.13,0.13}{#1}}
\newcommand\PYaa[1]{\textcolor[rgb]{0.50,0.50,0.50}{#1}}
\newcommand\PYaf[1]{\textcolor[rgb]{0.25,0.50,0.50}{\textit{#1}}}
\newcommand\PYag[1]{\textcolor[rgb]{0.40,0.40,0.40}{#1}}
\newcommand\PYad[1]{\textcolor[rgb]{0.73,0.13,0.13}{#1}}
\newcommand\PYae[1]{\textcolor[rgb]{0.40,0.40,0.40}{#1}}
\newcommand\PYaz[1]{\textcolor[rgb]{0.00,0.63,0.00}{#1}}
\newcommand\PYax[1]{\textcolor[rgb]{0.60,0.60,0.60}{\textbf{#1}}}
\newcommand\PYay[1]{\textcolor[rgb]{0.00,0.50,0.00}{\textbf{#1}}}
\newcommand\PYar[1]{\textcolor[rgb]{0.10,0.09,0.49}{#1}}
\newcommand\PYas[1]{\textcolor[rgb]{0.73,0.13,0.13}{\textit{#1}}}
\newcommand\PYap[1]{\textcolor[rgb]{0.00,0.50,0.00}{#1}}
\newcommand\PYaq[1]{\textcolor[rgb]{0.53,0.00,0.00}{#1}}
\newcommand\PYav[1]{\textcolor[rgb]{0.00,0.50,0.00}{\textbf{#1}}}
\newcommand\PYaw[1]{\textcolor[rgb]{0.40,0.40,0.40}{#1}}
\newcommand\PYat[1]{\textcolor[rgb]{0.10,0.09,0.49}{#1}}
\newcommand\PYau[1]{\textcolor[rgb]{0.40,0.40,0.40}{#1}}


\usetheme{Pittsburgh}
\author{Silesian Ruby User's Group}
\title{Ruby on Rails}
\setbeamercovered{transparent}

\begin{document}

\frame{\titlepage}

\section{Wstęp}
\begin{frame}
  \frametitle{Dlaczego web development?}
  \begin{itemize}
  \item przenośność
  \item niezależność od systemów operacyjnych i przeglądarek
  \item łatwe zarządzanie i utrzymanie
  \end{itemize}
\end{frame}

\begin{frame}
  \frametitle{Dlaczego Ruby on Rails?}
  \begin{itemize}
  \item przyjemność z programowania
  \item open source
  \item czytelność kodu
  \item propagowanie dobrych praktyk programistycznych
  \item szybkość programowania
  \item łatwość reagowania na zmiany
  \item niezależność od systemu operacyjnego
  \item niezależność od środowiska pracy (NetBeans, RadRails, Emacs,
    Vim, JEdit)
  \end{itemize}
\end{frame}

\section{Ruby}
\begin{frame}
  \frametitle{Ruby}
  \begin{itemize}
  \item 1995 rok, Yukihiro Matsumoto aka Matz
  \item inspirowany przez CLU, Eiffel, Lisp, Perl, Python, Smalltalk
  \item interpretowany
  \item bardzo wysokiego poziomu (VHLL)
  \item domknięcia (lambda/Proc)
  \end{itemize}
\end{frame}

\begin{frame}
  \frametitle{Ruby, c.d.}
  \begin{itemize}
  \item w pełni obiektowy
  \item prosta składnia, samokomentujący się kod
  \end{itemize}
\end{frame}

\begin{frame}
  \frametitle{Duck typing}
  \begin{itemize}
  \item rozpoznawanie typów na podstawie ich zachowania, a nie deklaracji
  \end{itemize}
  \begin{quote}
    If it walks like a duck and quacks like a duck, I would call it a
    duck.

    \hfill James Whitcomb Riley
  \end{quote}
\end{frame}

\begin{frame}
  \begin{itemize}
  \item - bloki i lambdy (closures) - wygodne przekazywanie funkcji
    jako parametrów + przykład
  \item domknięcia to dodatkowy powod, dla ktorego Rubisci nie boją
    się Javascriptu, patrz jQuery, Prototype, etc.
  \end{itemize}
\end{frame}

\begin{frame}
  \begin{itemize}
  \item dziedziczenie jednobazowe
  \item moduły - rodzaj dziedziczenia wielobazowego pozwalający
    włączyć gotową implementację zbioru metod do danej klasy (mixin)
  \item konwencja zakłada odejście od klasycznie wykładanego na
    uczelniach modelu OO, w którym lepszą ocenę dostaje osoba
    umiejąca wymyśleć większe drzewko dziedziczenia
  \item w Rubym dziedziczenia używa się znacznie oszczędniej i
    rzadziej niż np. w Javie
  \end{itemize}
\end{frame}

\begin{frame}
  \begin{itemize}
  \item otwarte klasy - monkeypatching + przykład
  \end{itemize}
\end{frame}

\begin{frame}[fragile]
  \begin{Verbatim}[commandchars=@\[\]]
@PYay[class] @PYaN[Module]
  @PYay[def] @PYaK[my_attr_accessor](@PYbe[*]symbols)
    symbols@PYbe[.]each @PYay[do] @PYbe[|]symbol@PYbe[|]
      @PYaX[module_eval] @PYaW["]@PYaW[def ]@PYbf[#{]symbol@PYbf[}]@PYaW[;]@PYaW["]\
                    @PYaW["]@PYaW[@at[]]@PYbf[#{]symbol@PYbf[}]@PYaW[;]@PYaW["]\
                  @PYaW["]@PYaW[end]@PYaW["]
      @PYaX[module_eval] @PYaW["]@PYaW[def ]@PYbf[#{]symbol@PYbf[}]@PYaW[=(value);]@PYaW["]\
                    @PYaW["]@PYaW[@at[]]@PYbf[#{]symbol@PYbf[}]@PYaW[ = value;]@PYaW["]\
                  @PYaW["]@PYaW[end]@PYaW["]
    @PYay[end]
  @PYay[end]
@PYay[end]
\end{Verbatim}

\end{frame}

\begin{frame}
  \begin{itemize}
  \item spójność
  \item nawiasy służą jedynie do zmiany priorytetu operatorów
  \item styl formatowania kodu nie jest narzucony z góry
  \item opcjonalny return (ostatnie wyrażenie jest wartością zwracaną
    przez metodę)
  \item bogata biblioteka standardowa + przykłady
  \item garbage collector
  \item przeciążanie operatorów
  \item liczby całkowite o dowolnych rozmiarach
  \item wyrażenia regularne wbudowane w składnię
  \end{itemize}
\end{frame}

\begin{frame}
  \begin{itemize}
  \item „pseudo-code that runs” - skupianie się na rozwiązaniu
    problemu, nie na języku, który jest zaprojektowany dla ludzi, nie
    dla maszyn
  \item Zasada najmniejszego zaskoczenia - Ruby jest intuicyjny
  \item Radość z programowania
  \item Wolność wyboru (jak w Perlu, przeciwnie niż w Pythonie) -
    TIMTOWTDI, ale konwencje pozwalają na uniknięcie typowego
    „perlarstwa”
  \end{itemize}
\end{frame}

\section{Ruby on Rails}
\begin{frame}
  \frametitle{Ruby on Rails}
  \begin{itemize}
  \item DHH, 2004/2005
  \end{itemize}
\end{frame}

\begin{frame}
  \frametitle{Full Stack Framework w MVC}
  architektura
\end{frame}

\begin{frame}
  \frametitle{ActiveRecord}
  \begin{itemize}
  \item jeden z ORMów możliwych do użycia (np. DataMapper)
  \item implementacja wzorca
  \end{itemize}
\end{frame}

\begin{frame}
  \begin{quote}
    I have never seen an Active Record implementation as complete or as useful as Rails

    \hfill Martin Fowler
  \end{quote}
\end{frame}

\begin{frame}
  \begin{itemize}
  \item brak XML
  \item migracje
  \item proste tworzenie asocjacji
  \item walidatory
  \item callbacki
  \item transakcje
  \item STI
  \end{itemize}
\end{frame}

\begin{frame}
  przykłady do AR
\end{frame}

\begin{frame}
  \frametitle{ActionPack}
  On rails from request to response
\end{frame}

\begin{frame}
  \frametitle{ActiveResource}
  mapowanie RESTowych zasobów jako modele
\end{frame}

\begin{frame}
  \frametitle{ActionMailer}
\end{frame}

\begin{frame}
  \frametitle{ActiveSupport}
\end{frame}

\begin{frame}
  serwery HTTP: Webrick, Mongrel, Thin, Ebb, Passenger(Apache)
\end{frame}

\begin{frame}
  Przykłady wdrożeń
\end{frame}

\begin{frame}
  \frametitle{Rails way}
  \begin{itemize}
  \item Jeden sposób na zaprojektowanie aplikacji webowej
  \item Convention over Configuration
  \item Don't Repeat Yourself
  \item Fat model, thin controller w MVC
  \end{itemize}
\end{frame}

\section{Ruby - dodatki}
\begin{frame}
  \frametitle{RubyGems}
  system paczek: gem install rails
\end{frame}

\begin{frame}
  \frametitle{Rake}
  \begin{itemize}
  \item Rake - Ruby Make
  \item przykłady
  \end{itemize}
\end{frame}

\begin{frame}
  \frametitle{RSpec}
  \begin{itemize}
  \item framework BDD
  \item przykłady
  \end{itemize}
\end{frame}

\section{Ruby - rozwój}
\begin{frame}
  \frametitle{Ruby - rozwój}
  \begin{itemize}
  \item implementacje: MRI, Ruby Enterprise, JRuby, IronRuby
  \item maszyny wirtualne: MagLev, Rubinius, YARV
  \item Github
  \end{itemize}
\end{frame}

\section{Ruby - podsumowanie}
\begin{frame}
  \frametitle{Jak zacząć}
  \begin{itemize}
  \item instant-rails
  \item github
  \item opensourcerails.com
  \item heroku.com
  \end{itemize}
\end{frame}

\begin{frame}
  \frametitle{Po 3 latach z Ruby - co się zmienia}
  \begin{itemize}
  \item Ruby jako pierwszy pozwolił mi czerpać radość z
    programowania. Wcześniej były Java, w C/C++ i inne
  \item Pragmatyzm. Prosty, czytelny kod. Mało kodu. Brak
    komentarzy. Więcej czasu na życie.
  \item Nie wrócę do już do języków ze statycznym typowaniem. Czytałem
    Bruca Eckela kiedy mówił, że statyczne typowanie to przyszłość. Po
    latach zmienił zdanie. To samo dotyczy deklarowania wyrzucanych
    wyjątków (patrz Java).
  \item Większy dystans i szersze spojrzenie na rzemiosło programowania.
  \item 1 nowy język programowania co roku (ObjC, Erlang, Lisp, ...)
  \end{itemize}
\end{frame}

\end{document}
