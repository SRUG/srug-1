%%
%% presentation.tex
%%
%% Made by Jakub Kuźma
%% Login   <kuba@jah.pl>
%%
%% Started on  Fri Oct 24 19:44:04 2008 Jakub Kuźma
%% Last update Fri Oct 24 19:44:04 2008 Jakub Kuźma
%%

\documentclass[12t]{beamer}
\usepackage[utf8]{inputenc}
\usepackage{polski}
\usepackage[polish]{babel}
\usepackage{graphicx}
\usepackage{color}
\usepackage{listings}
\usetheme{Pittsburgh}
\author{Silesian Ruby User's Group}
\title{Ruby on Rails}
\setbeamercovered{transparent}

\begin{document}

\frame{\titlepage}

\section{Wstęp}
\begin{frame}
  \frametitle{Dlaczego web development?}
  \begin{itemize}
  \item przenośność
  \item niezależność od systemów operacyjnych i przeglądarek
  \item łatwe zarządzanie i utrzymanie
  \end{itemize}
\end{frame}

\begin{frame}
  \frametitle{Dlaczego Ruby on Rails?}
  \begin{itemize}
  \item przyjemność z programowania
  \item open source
  \item czytelność kodu
  \item propagowanie dobrych praktyk programistycznych
  \item szybkość programowania
  \item łatwość reagowania na zmiany
  \item niezależność od systemu operacyjnego
  \item niezależność od środowiska pracy (NetBeans, RadRails, Emacs,
    Vim, JEdit)
  \end{itemize}
\end{frame}

\section{Ruby}
\begin{frame}
  \frametitle{Ruby}
  \begin{itemize}
  \item 1995 rok, Yukihiro Matsumoto aka Matz
  \item inspirowany przez CLU, Eiffel, Lisp, Perl, Python, Smalltalk
  \item interpretowany
  \item bardzo wysokiego poziomu (VHLL)
  \item domknięcia (lambda/Proc)
  \end{itemize}
\end{frame}

\begin{frame}
  \frametitle{Ruby, c.d.}
  \begin{itemize}
  \item w pełni obiektowy
  \item prosta składnia, samokomentujący się kod
  \end{itemize}
\end{frame}

\begin{frame}
  \frametitle{Duck typing}
  \begin{itemize}
  \item rozpoznawanie typów na podstawie ich zachowania, a nie deklaracji
  \end{itemize}
  \begin{quote}
    If it walks like a duck and quacks like a duck, I would call it a
    duck.

    \hfill James Whitcomb Riley
  \end{quote}
\end{frame}

\begin{frame}
  \begin{itemize}
  \item - bloki i lambdy (closures) - wygodne przekazywanie funkcji
    jako parametrów + przykład
  \item domknięcia to dodatkowy powod, dla ktorego Rubisci nie boją
    się Javascriptu, patrz jQuery, Prototype, etc.
  \end{itemize}
\end{frame}

\begin{frame}
  \begin{itemize}
  \item dziedziczenie jednobazowe
  \item moduły - rodzaj dziedziczenia wielobazowego pozwalający
    włączyć gotową implementację zbioru metod do danej klasy (mixin)
  \item konwencja zakłada odejście od klasycznie wykładanego na
    uczelniach modelu OO, w którym lepszą ocenę dostaje osoba
    umiejąca wymyśleć większe drzewko dziedziczenia
  \item w Rubym dziedziczenia używa się znacznie oszczędniej i
    rzadziej niż np. w Javie
  \end{itemize}
\end{frame}

\begin{frame}
  \begin{itemize}
  \item otwarte klasy - monkeypatching + przykład
  \end{itemize}
\end{frame}

\begin{frame}
  \begin{itemize}
  \item spójność
  \item nawiasy służą jedynie do zmiany priorytetu operatorów
  \item styl formatowania kodu nie jest narzucony z góry
  \item opcjonalny return (ostatnie wyrażenie jest wartością zwracaną
    przez metodę)
  \item bogata biblioteka standardowa + przykłady
  \item garbage collector
  \item przeciążanie operatorów
  \item liczby całkowite o dowolnych rozmiarach
  \item wyrażenia regularne wbudowane w składnię
  \end{itemize}
\end{frame}

\begin{frame}
  \begin{itemize}
  \item „pseudo-code that runs” - skupianie się na rozwiązaniu
    problemu, nie na języku, który jest zaprojektowany dla ludzi, nie
    dla maszyn
  \item Zasada najmniejszego zaskoczenia - Ruby jest intuicyjny
  \item Radość z programowania
  \item Wolność wyboru (jak w Perlu, przeciwnie niż w Pythonie) -
    TIMTOWTDI, ale konwencje pozwalają na uniknięcie typowego
    „perlarstwa”
  \end{itemize}
\end{frame}

\section{Ruby on Rails}
\begin{frame}
  \centering Ruby on Rails
\end{frame}

\begin{frame}
  \frametitle{Ruby on Rails}
  \begin{itemize}
  \item David Heinemeier Hansson, 2004 r.
  \item Rails to w pełni wyposażone środowisko do tworzenia aplikacji internetowych opartych o bazy danych zgodnie ze wzorcem MVC
  \end{itemize}
\end{frame}

\begin{frame}
  \frametitle{Z czego składa się Rails?}
  \begin{itemize}
  \item ActiveRecord
  \item ActionPack
  \item ActiveResource
  \item ActionMailer
  \item ActiveSupport
  \end{itemize}
\end{frame}

\begin{frame}
  \frametitle{ActiveRecord}
  \begin{itemize}
  \item domyślny ORM dla Rails
  \item są też inne opcje (np. DataMapper)
  \item implementacja wzorca Active Record
  \note{M.Fowler jest autorem książek i znanym wykładowcą z tematyki architektury oprogramowania, specjalizującym się w analizie obiektowej i projektowaniu, UML, wzorcach projektowych, metodykach zwinnych, w tym Programowania ekstremalnego}
  \end{itemize}
\end{frame}

\begin{frame}
  \begin{quote}
    I have never seen an Active Record implementation as complete or as useful as Rails

    \hfill Martin Fowler
  \end{quote}
\end{frame}

\begin{frame}
  \frametitle{ActiveRecord - co dostajemy?}
  \begin{itemize}
  \item brak XML'a
  \item migracje bazy danych
  \item proste tworzenie asocjacji
  \item walidatory
  \item callbacki
  \item transakcje
  \item masę innych rzeczy...
  \end{itemize}
\end{frame}

\begin{frame}
  \frametitle{ActiveRecord - Brak XML}
  \begin{itemize}
  \item nie wymaga konfiguracji za pomocą setek linii XML'a
  \item mapowanie tabel na modele - konwencje nad konfiguracją
  \end{itemize}
\end{frame}

\begin{frame}
  \frametitle{ActiveRecord - Brak XML - przykład}

\end{frame}

\begin{frame}
  \frametitle{ActiveRecord - Migracje}
  \begin{itemize}
  \item proste w użyciu wersjonowanie schematu bazy, historia zmian
  \item Ruby zamiast SQL'a
  \item praca w zespołach
  \item synchronizacja
  \end{itemize}
\end{frame}

\begin{frame}
  \frametitle{ActiveRecord - Migracje - przykład}
  kod
\end{frame}

\begin{frame}
  \frametitle{ActiveRecord - Asocjacje}
  \begin{itemize}
  \item mapowanie pomiędzy obiektami ActiveRecord
  \item wyrażają relację takie jak 'Użytkownik ma wiele Projektów' czy 'Produkt należy do Kategorii'
  \item oparte na metaprogramowaniu
  \item wczesne ładowanie obiektów
  \item połączenia wiele-do-wielu
  \end{itemize}
\end{frame}

\begin{frame}
  \frametitle{ActiveRecord - Asocjacje - przykład}
\end{frame}

\begin{frame}
  \frametitle{ActiveRecord - Walidatory}
  \begin{itemize}
  \item gwarantują poprawne dane w bazie
  \item przeniesienie walidacji z poziomu bazy danych do mapowanych obiektów
  \item można walidować: format, długość, obecność, unikalność, powiązane obiekty ...
  \item łatwe dodawanie własnych walidatorów
  \item metaprogramowanie :)
  \end{itemize}
\end{frame}

\begin{frame}
  \frametitle{ActiveRecord - Walidatory - przykład}
  kod
\end{frame}

\begin{frame}
  \frametitle{ActiveRecord - Callbacki}
  \begin{itemize}
  \item wyzwalanie logiki przed lub po zmianach stanu obiektów
  \item manipulacja atrybutami obiektu przed jego utworzeniem, zapisem, usunięciem lub walidacją
  \item przeniesienie logiki z kontrolera do modelu
  \end{itemize}
\end{frame}

\begin{frame}
  \frametitle{ActiveRecord - Callbacki - przykład}
  kod
\end{frame}

\begin{frame}
  \frametitle{ActiveRecord - Transakcje}
  \begin{itemize}
  \item bloki kodu, w których gwarantowana jest atomowość wszystkich operacji na bazie danych
  \item różne modele w jednej transakcji
  \end{itemize}
\end{frame}

\begin{frame}
  \frametitle{ActiveRecord - Transakcje - przykład}
  kod
\end{frame}

\begin{frame}
  \frametitle{ActiveRecord}
  \begin{itemize}
  \item Single Table Inheritance
  \item asocjacje polimorficzne
  \item cachowanie
  \item acts\_as: state\_machine, taggable, nested\_set, commentable, dictionary, geocodable ...
  \end{itemize}
\end{frame}

\begin{frame}
  \frametitle{ActionPack}
  \begin{itemize}
  \item podział odpowiedzi aplikacji na dwie części:
    \begin{itemize}
    \item ActionController
    \item ActionView
    \end{itemize}
  \end{itemize}
\end{frame}

\begin{frame}
  \frametitle{ActionPack::ActionController}
  \begin{itemize}
  \item akcje w kontrolerze
  \item akcja renderuje daną stronę, wywołuje inną akcję bądź zwracają zasoby (REST)
  \item sesje
  \item flash
  \item callbacki
  \item akcje CRUD
  \end{itemize}
\end{frame}

\begin{frame}
  \frametitle{ActionPack::ActionController}
  przykłady ActionController
\end{frame}

\begin{frame}
  \frametitle{ActionPack::ActionView}
  \begin{itemize}
  \item szablony w widokach: ERb, Haml, Liquid i inne
  \item partiale
  \item helpery
  \end{itemize}
\end{frame}

\begin{frame}
  \frametitle{ActionPack::ActionView}
  przykłady ActionView
\end{frame}

\begin{frame}
  \frametitle{ActiveResource}
  mapowanie RESTowych zasobów jako modele
\end{frame}

\begin{frame}
  \frametitle{ActionMailer}
  \begin{itemize}
  \item wbudowany system do rozsyłania i odbierania poczty email
  \item sendmail, IMAP
  \end{itemize}
\end{frame}

\begin{frame}
  \frametitle{ActiveSupport}
\end{frame}

\begin{frame}
  serwery HTTP: Webrick, Mongrel, Thin, Ebb, Passenger(Apache)
\end{frame}

\begin{frame}
  Przykłady wdrożeń
\end{frame}

\begin{frame}
  \frametitle{Rails way}
  \begin{itemize}
  \item Jeden sposób na zaprojektowanie aplikacji webowej
  \item Convention over Configuration
  \item Don't Repeat Yourself
  \item Fat model, thin controller w MVC
  \end{itemize}
\end{frame}

\section{Ruby - dodatki}
\begin{frame}
  \frametitle{RubyGems}
  system paczek: gem install rails
\end{frame}

\begin{frame}
  \frametitle{Rake}
  \begin{itemize}
  \item Rake - Ruby Make
  \item przykłady
  \end{itemize}
\end{frame}

\begin{frame}
  \frametitle{RSpec}
  \begin{itemize}
  \item framework BDD
  \item przykłady
  \end{itemize}
\end{frame}

\section{Ruby - rozwój}
\begin{frame}
  \frametitle{Ruby - rozwój}
  \begin{itemize}
  \item implementacje: MRI, Ruby Enterprise, JRuby, IronRuby
  \item maszyny wirtualne: MagLev, Rubinius, YARV
  \item Github
  \end{itemize}
\end{frame}

\section{Ruby - podsumowanie}
\begin{frame}
  \frametitle{Jak zacząć}
  \begin{itemize}
  \item instant-rails
  \item github
  \item opensourcerails.com
  \item heroku.com
  \end{itemize}
\end{frame}

\begin{frame}
  \frametitle{Po 3 latach z Ruby - co się zmienia}
  \begin{itemize}
  \item Ruby jako pierwszy pozwolił mi czerpać radość z
    programowania. Wcześniej były Java, w C/C++ i inne
  \item Pragmatyzm. Prosty, czytelny kod. Mało kodu. Brak
    komentarzy. Więcej czasu na życie.
  \item Nie wrócę do już do języków ze statycznym typowaniem. Czytałem
    Bruca Eckela kiedy mówił, że statyczne typowanie to przyszłość. Po
    latach zmienił zdanie. To samo dotyczy deklarowania wyrzucanych
    wyjątków (patrz Java).
  \item Większy dystans i szersze spojrzenie na rzemiosło programowania.
  \item 1 nowy język programowania co roku (ObjC, Erlang, Lisp, ...)
  \end{itemize}
\end{frame}

\end{document}
